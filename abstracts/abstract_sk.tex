\chapter*{Abstrakt}\label{chap:abstract_sk}

Počas astronomických pozorovaní uskutočnených pomocou 70 cm
ďalekohľadu určeného na pozorovanie vesmírneho odpadu sú získavané snímky
konkrétnej časti nočnej oblohy. Každý pixel
takejto snímky je reprezentovaný tromi údajmi – pozíciou na horizontálnej osi
x, pozíciou na vertikálnej osi y a intenzitou, ktorá nadobúda hodnoty od 0 až
po 65 536. Nulová, respektíve veľmi
nízka intenzita znamená, že na danej časti nočnej oblohy, ktorá korešponduje
s x a y súradnicami na snímke, nebol zaznamenaný žiaden objekt. Naopak,
intenzity pohybujúce sa v rádovo tisíckach
hodnôt a vyššie indikujú prítomnosť nejakého objektu, ako napríklad hviezdy,
objektov slnečnej
sústavy, vesmírneho odpadu, alebo aj šumu elektrického prúdu, pozadia oblohy
a iných artefaktov.
Úlohou študenta bude zoznámiť sa s už existujúcimi algoritmami používanými
na riešenie
danej problematiky, naštudovať si poskytnutú odbornú literatúru, a navrhnúť
a otestovať vlastný
algoritmus.


\bigskip

\todo[inline, color=red!50]{Abstrakt}



\bigskip

\noindent \textbf{Kľúčové slová:} segmentácia, sledovanie objektu, spracovanie astronomických snímok
\vfill\eject 
