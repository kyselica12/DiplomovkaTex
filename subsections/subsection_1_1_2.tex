\subsection{Základné typy grafov}

Orientovaným grafom označujeme graf, ktoré hrany sú orientované, teda majú zadaný smer. Takéto hrany označujeme šípkou na hrane, pričom smer šípky vyznačuje smer orientácie. Pri textovom zápise množiny hrán sa orientovaná hrana zapisuje ako dvojica vrcholov, pričom prvý vrchol je ten, z ktorého hrana vychádza, a druhý vrchol je ten, do ktorého hrana vchádza.\\

Neorientovaný graf obsahuje neorientované hrany. Z toho vyplýva, že hrany nemajú žiaden zadaný smer a teda v ich reprezentácii ako množiny dvojíc pre každú dvojicu platí, že na poradí uzlov v danej dvojici nezáleží.\\

Graf nazývame jednoduchým, ak neobsahuje žiadne slučky ani viacnásobné hrany. V prípade, že graf neobsahuje viacnásobné hrany, tak hovoríme prostom grafe. Multigraf naopak predstavuje typ grafu, ktorý obsahuje viacnásobné hrany. Každý graf, ktorá obsahuje slučky môžme nazývať aj ako pseudograf.
